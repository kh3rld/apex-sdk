% Apex SDK Research Paper Template
% Version 1.0 - December 2025

\documentclass[11pt,a4paper]{article}

% Packages
\usepackage[utf8]{inputenc}
\usepackage[english]{babel}
\usepackage{amsmath,amssymb,amsthm}
\usepackage{graphicx}
\usepackage{hyperref}
\usepackage{listings}
\usepackage{xcolor}
\usepackage{geometry}
\usepackage{fancyhdr}
\usepackage{abstract}
\usepackage{titlesec}

% Page geometry
\geometry{
    a4paper,
    left=25mm,
    right=25mm,
    top=30mm,
    bottom=30mm
}

% Header and footer
\pagestyle{fancy}
\fancyhf{}
\fancyhead[L]{\small\itshape Your Paper Title}
\fancyhead[R]{\small Apex SDK Research Papers}
\fancyfoot[C]{\thepage}

% Hyperlinks
\hypersetup{
    colorlinks=true,
    linkcolor=blue,
    filecolor=magenta,
    urlcolor=cyan,
    citecolor=blue
}

% Code listing style for Rust
\definecolor{codebackground}{RGB}{245,245,245}
\definecolor{codekeyword}{RGB}{0,102,204}
\definecolor{codecomment}{RGB}{0,128,0}
\definecolor{codestring}{RGB}{163,21,21}

\lstdefinestyle{rust}{
    language=C++,  % Approximation for Rust
    backgroundcolor=\color{codebackground},
    commentstyle=\color{codecomment},
    keywordstyle=\color{codekeyword}\bfseries,
    numberstyle=\tiny\color{gray},
    stringstyle=\color{codestring},
    basicstyle=\ttfamily\small,
    breakatwhitespace=false,
    breaklines=true,
    captionpos=b,
    keepspaces=true,
    numbers=left,
    numbersep=5pt,
    showspaces=false,
    showstringspaces=false,
    showtabs=false,
    tabsize=2,
    frame=single,
    rulecolor=\color{black}
}

\lstset{style=rust}

% Title information
\title{\LARGE\bfseries Your Research Paper Title:\\
\large A Subtitle if Needed}

\author{
    Author Name\textsuperscript{1}, Co-Author Name\textsuperscript{2} \\
    \small\textsuperscript{1}Your Organization/University \\
    \small\textsuperscript{2}Co-Author Organization \\
    \small\texttt{your.email@example.com}, \texttt{coauthor@example.com}
}

\date{December 2025}

% Document
\begin{document}

\maketitle

\begin{abstract}
Write a concise abstract (200-300 words) that summarizes:
\begin{itemize}
    \item The problem you're addressing
    \item Your approach or methodology
    \item Key findings or results
    \item Implications or significance
\end{itemize}

This abstract should be self-contained and give readers a complete overview of your work. Avoid citations and abbreviations if possible.
\end{abstract}

\noindent\textbf{Keywords:} Apex SDK, blockchain, cross-chain, Rust, your-keyword-1, your-keyword-2

\tableofcontents
\newpage

\section{Introduction}

Introduce the problem domain, motivation, and context for your research. Explain why this work is important for the Apex SDK ecosystem or blockchain development in general.

\subsection{Background}

Provide necessary background information about:
\begin{itemize}
    \item Apex SDK architecture and capabilities
    \item Relevant blockchain technologies (Substrate, EVM, etc.)
    \item The specific problem you're addressing
\end{itemize}

\subsection{Motivation}

Explain what motivated this research. What gap does it fill? What problem does it solve?

\subsection{Contributions}

List your main contributions clearly:
\begin{enumerate}
    \item First contribution
    \item Second contribution
    \item Third contribution
\end{enumerate}

\section{Related Work}

Review existing research and implementations related to your work. Compare and contrast with your approach.

\subsection{Existing Solutions}

Discuss current approaches and their limitations.

\subsection{Positioning Your Work}

Explain how your work differs from and improves upon existing solutions.

\section{Methodology}

\subsection{Approach}

Describe your technical approach in detail. Include:
\begin{itemize}
    \item System architecture
    \item Design decisions
    \item Implementation choices
\end{itemize}

\subsection{Implementation}

Provide implementation details with code examples:

\begin{lstlisting}[caption=Example Rust code using Apex SDK]
use apex_sdk::{Client, Config};
use apex_sdk::substrate::SubstrateClient;

async fn example() -> Result<(), Box<dyn std::error::Error>> {
    let config = Config::new("wss://rpc.polkadot.io");
    let client = SubstrateClient::connect(config).await?;

    // Your implementation here

    Ok(())
}
\end{lstlisting}

\subsection{Tools and Environment}

Document your development environment:
\begin{itemize}
    \item Apex SDK version: 0.1.4
    \item Rust version: 1.85+
    \item Operating system: Linux/macOS/Windows
    \item Additional dependencies
\end{itemize}

\section{Experimental Setup}

\subsection{Test Environment}

Describe your testing environment and methodology.

\subsection{Benchmarking Methodology}

Explain how you measured performance, correctness, or other metrics.

\subsection{Datasets}

If applicable, describe any datasets used in your evaluation.

\section{Results}

Present your findings with tables, charts, and detailed analysis.

\subsection{Performance Results}

\begin{table}[h]
\centering
\begin{tabular}{|l|c|c|}
\hline
\textbf{Metric} & \textbf{Value} & \textbf{Notes} \\
\hline
Transaction Throughput & 850 TPS & Average \\
Latency & 120ms & p95 \\
Memory Usage & 45 MB & Idle \\
\hline
\end{tabular}
\caption{Performance benchmarks}
\label{tab:performance}
\end{table}

\subsection{Comparative Analysis}

Compare your results with baseline or competing approaches.

\section{Discussion}

\subsection{Analysis}

Interpret your results and explain their significance.

\subsection{Limitations}

Honestly discuss the limitations of your work.

\subsection{Future Work}

Suggest directions for future research and improvements.

\section{Case Study (Optional)}

If applicable, present a real-world application of your work.

\section{Conclusion}

Summarize your work, main findings, and their implications for the Apex SDK ecosystem.

\section*{Acknowledgments}

Acknowledge contributors, funding sources, or supporting organizations.

\begin{thebibliography}{9}

\bibitem{apexsdk}
Apex SDK Team,
\textit{Apex SDK: Unified Rust SDK for Substrate \& EVM},
\url{https://apexsdk.dev},
2025.

\bibitem{substrate}
Parity Technologies,
\textit{Substrate: A Blockchain Framework},
\url{https://substrate.io},
2025.

\bibitem{rust}
The Rust Programming Language,
\url{https://www.rust-lang.org},
2025.

% Add your references here

\end{thebibliography}

\appendix

\section{Additional Code Examples}

Include longer code listings or additional examples here.

\section{Detailed Benchmarks}

Include comprehensive benchmark data that would clutter the main text.

\end{document}
